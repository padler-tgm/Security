\documentclass[11pt]{scrartcl}
\usepackage{ucs}
\usepackage{float}
\usepackage[utf8x]{inputenc}
\usepackage{ngerman}
\usepackage{amsmath,amssymb,amstext}
\usepackage{graphicx}
\usepackage{tabularx}
\usepackage[square]{natbib}
\usepackage[justification=RaggedRight, singlelinecheck=false]{caption} 
\usepackage{fancyhdr}

\pagestyle{fancy}
\lfoot{Philipp Adler}
\rfoot{\today{}}

\title{Sicherheit}
\author{Philipp Adler}
\date{\today{}}

\begin{document}

\maketitle
\tableofcontents
\pagebreak

\section{Grundlagen Securityverfahren}
\label{sec:basics-security-process}
% Moderne Kryptographie 2

\subsection{Verschlüsselungsarten}
\label{sec:ciphering types}

\subsection{Kommunikationsszenarien}
\label{sec:communication-scenarios}

\subsection{Sicherheitsziele}
\label{sec:security goals}

\subsection{Bedrohungsszenarien}
\label{sec:threat scenarios}


\section{Symmetrische Verschlüsselung}
\label{sec:symetric-ciphering}
Die symmetrische Verschlüsselung reicht bis in die Antike. Damals wussten nur die Adressierten von dem Geheimnis, nach welchem Verfahren die Botschaft verschlüsselt wurde. Cäsar zum Beispiel verschob jeden Buchstaben um 4 Stellen. So wurde aus Hallo Kdoor. Aus diesem Verschlüsselungsalgorithmus entstanden zum einen Blockchiffren, auf den ich in dem folgenden Kapitel näher eingehen werde und die Stromchiffren.\cite{1}

\subsection{Blockchiffre}
\label{sec:blockchiffre}
% Praktische Kryptographie unter Linux 2.3
% Sicherheit und Kryptographie im Internet 1.4.2
Blockchiffren teilen die Nachricht, die verschlüsselt werden soll, in eine fixe Anzahl an Blöcken, die entweder 64 oder 128 Bit groß sind. Typische, bekannte Blockchiffre sind Data Encryption Standard, Advanced Encryption Standard und International Data Encryption Algorithm.\cite{1}
\subsubsection{Der Data Encryption Standard - DES}
\label{sec:data-encryotion-standard}
"DES wurde 1977 vom amerikanischen 'National Institute of Standards and Technologies (NIST)' veröffentlicht."\cite{1} Bei diesem Verfahren wird eine Blocklänge von 64 Bit und ein DES-Schlüssel von 56 Bits plus 8 "Parity Check Bits"\cite{1} eingesetzt. Die ersten 56 Bits werden immer zufällig generiert, wobei die letzten 8 Bits dafür sorgen, dass keine Übertragungsfehler auftreten. Da 56 Bit zufällig sind, können daraus 2\textsuperscript{56} Schlüsseln erzeugt werden.
\textbf{Das Schema}
Beim Verschlüsselungsverfahren werden aus dem Klartext Blöcke alle jeweils mit einer Länge von 64 Bit erzeugt. Dieser Block wird dann nochmals zerlegt, sodass daraus 2 mal 32 Bit Blöcke entstehen. Der Data Encryption Standard besteht aus 16 Runden. In jeder Runde wird auf die rechte Hälfte ein Verschlüsselungsalgorithmus \textit{f} angewendet. Das daraus resultierende Ergebnis wird bitweise mit einem XOR-Gatter mit der linken Hälfte verknüpft und "bildet die rechte Seite der neuen Runde"\cite{2}.

wird die linke Hälft mit der Rechten vertauscht und anschließend wird die Linke mit einem Verschlüsselungsalgorithmus \textit{f} bitweise mit einem XOR-Gatter  verknüpft. In dieser Funktionen befinden sich zum einen der Parameter \textit{k{\footnotesize i}} und zum anderem\textit{R{\footnotesize i}}. 
% Kryptografie in Theorie und Praxis 7.6
% Kryptograhie und IT-Sicherheit 2.2
% Verteilte Systeme S.427

\subsubsection{Der Advanced Encryption Standard - AES}
\label{sec:advanced-encryotion-standard}
% Kryptografie in Theorie und Praxis 7.7
% Kryptograhie und IT-Sicherheit 2.6
% Moderne Kryptographie 4.5

\subsubsection{IDEA (International Data Encryption Algorithm)}
\label{sec:international-data-encryption-algorithm}
% Kryptograhie und IT-Sicherheit 2.3

\subsection{Der RSA-Algorithmus}
\label{sec:rsa-algorithmus}
% Praktische Kryptographie unter Linux 2.4.3
% Kryptografie in Theorie und Praxis 10
% Kryptologie 5.3.4
% Kryptograhie und IT-Sicherheit 4.2
% Moderne Kryptographie 6.4
% Sicherheit und Kryptographie im Internet 1.5.4
% Verteilte Systeme S.430

\subsubsection{Schlüsselerzeugung}
\label{sec:key-generation}

\subsubsection{Verschlüsseln}
\label{sec:rsa-encrypt}

\subsubsection{Entschlüsseln}
\label{sec:rsa-decrypt}

\section{SSL/TLS-Protokoll}
\label{sec:ssl/tls-protocol}
% Praktische Kryptographie unter Linux 8.1
% Kryptograhie und IT-Sicherheit 6.3
% Sicherheit und Kryptographie im Internet 7.7/7.9
% Verteilte Systeme S.626

\subsection{SSL/TLS Grundlagen}
\label{sec:ssl-tls-basics}

\subsection{SSL/TLS im Protokollstapel}
\label{sec:ssl-tls-protocolstack}

\subsection{Client-Server-Kommunikation}
\label{sec:client-server-communication}



\subsection{Das SSL-Handshake}
\label{sec:ssl-handshake}

\subsection{TLS asymetrisch}
\label{sec:tls-asymetric}

\subsection{TLS symetrisch}
\label{sec:tls-symetric}

\subsection{TLS hybrid}
\label{sec:tls-hybrid}

\subsection{TLS Algorithmen}
\label{sec:tls-algorithmen}


\section{Schwierigkeiten bei Software}
\label{sec:sw-trouble}

\subsection{Buffer Ovrflow}
\label{sec:buffer-overflow}
% Improving Intrusion Detection Systems 4.3
% Network Intrusion Detection and Prevention 1.3.1

\subsection{OpenSSL}
\label{sec:openssl}
% Praktische Kryptographie unter Linux 8.2
% Sicherheit und Kryptographie im Internet 7.10

\newpage
\listoffigures
\addcontentsline{toc}{section}{Abbildungsverzeichnis}
\addcontentsline{toc}{section}{Literaturverzeichnis}
\mbox{}
\nocite{*}
\bibliographystyle{unsrt}
\bibliography{literatur}

\end{document}