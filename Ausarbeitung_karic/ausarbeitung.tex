\documentclass[11pt]{scrartcl}
\usepackage{ucs}
\usepackage{float}
\usepackage[utf8x]{inputenc}
\usepackage{ngerman}
\usepackage{amsmath,amssymb,amstext}
\usepackage{graphicx}
\usepackage{tabularx}
\usepackage[square]{natbib}
\usepackage[justification=RaggedRight, singlelinecheck=false]{caption} 
\usepackage{fancyhdr}

\pagestyle{fancy}
\lfoot{Adin Karic}
\rfoot{\today{}}

\title{Sicherheit}
\author{Adin Karic}
\date{\today{}}

\begin{document}

\maketitle
\tableofcontents
\pagebreak

\section{Grundlagen Securityverfahren}
\label{sec:basics-security-process}
% Moderne Kryptographie 2

\subsection{Verschlüsselungsarten}
\label{sec:ciphering types}
%Angewandte Kryptographie

\subsection{Kommunikationsszenarien}
\label{sec:communication-scenarios}
%Kryptographie,Wätjen

\subsection{Sicherheitsziele}
\label{sec:security goals}

\subsection{Bedrohungsszenarien}
\label{sec:threat scenarios}

\section{Intrusion Detection Systeme}
\label{sec:IDS}

\subsection{Network-based IDS}
\label{sec:nw-based IDS}

\subsection{Host-based IDS}
\label{sec:host based IDS}

\section{Intrusion Prevention Systeme}
\label{sec:intrusion prevention systems}
%HIPS
%NIPS

\section{Honey Pot Systeme}
\label{sec:honey pot systems}

\section{Application Firewall}
\label{sec:application firewall}

\subsection{Gründe für eine Application Firewall}
\label{sec:reasons AF}

\subsection{Host-based Application Firewall}
\label{sec:host-based application firewall}

\subsection{Network-based Application Firewall}
\label{sec:network-based application firewall}

\section{Literaturverzeichnis}
\label{sec:bibliography}

%heuristics?

\newpage
\mbox{}
\nocite{*}
\bibliographystyle{unsrt}
\bibliography{literatur}

\end{document}